\subsection{Pipeline Evaluation}\label{pipeline-evaluation}

\subsubsection{Method and Corpus}\label{method-and-corpus}

400 reports from the University Hospital Zurich were picked at random
across around 30 document types (mainly discharge reports) giving rise
to around 2.7 Mio tokens. These reports then got annotated using a
prelimninary version of the pipeline. A medical student then went
through these annotations in GATE Developer and complemented/corrected
the existing ones. In a last step, the annotations got checked by a
member of the USZ staff. Then, the annotations of the mature version of
the pipeline was compared to the verified annotations (``goldstandard'')
and evaluated.

The corpus was split into two parts with 200 reports each. The first
part was heavily used for development and tuning (``training set'')
whereas the second was only used for evaluation (``validation set''). No
evaluation on a strictly unseen test set was performed.

A bit more details can be found in Section~\ref{evaluation-guide}.

\subsubsection{Results}\label{results}

The following tables contain Precision/Recall by annotation types
comparing the pipeline output with the annotations in the goldstandard.
The \texttt{Recall} and \texttt{Precision} columns refer to exact
matches of annotations, the columns \texttt{Recall\ Lenient} and
\texttt{Precision\ Lenient} also include partial matches, for instance,
for the name \texttt{Hanna\ Meier\ Huber}, the pipeline only annotates
\texttt{Hanna\ Meier}.

\paragraph{Entire Goldstandard Corpus (Parts I +
II):}\label{entire-goldstandard-corpus-parts-i-ii}

\begin{longtable}[]{@{}llllll@{}}
\toprule\noalign{}
Type & Matches & Recall & Recall Lenient & Precision & Precision
Lenient \\
\midrule\noalign{}
\endhead
\bottomrule\noalign{}
\endlastfoot
Age & 716 & 94.584 & 94.584 & 95.086 & 95.086 \\
Contact & 4805 & 99.154 & 99.505 & 99.154 & 99.505 \\
Date & 47223 & 99.145 & 99.397 & 98.574 & 98.825 \\
ID & 187 & 79.915 & 85.897 & 3.166 & 3.403 \\
Location & 31427 & 85.300 & 91.415 & 84.309 & 90.353 \\
Name & 17422 & 98.787 & 99.484 & 94.561 & 95.229 \\
Occupation & 290 & 55.238 & 63.619 & 60.291 & 69.439 \\
\end{longtable}

\paragraph{Goldstandard Part I}\label{goldstandard-part-i}

\begin{longtable}[]{@{}llllll@{}}
\toprule\noalign{}
Type & Matches & Recall & Recall Lenient & Precision & Precision
Lenient \\
\midrule\noalign{}
\endhead
\bottomrule\noalign{}
\endlastfoot
Age & 361 & 96.524 & 96.524 & 93.282 & 93.282 \\
Contact & 2384 & 99.375 & 99.625 & 98.962 & 99.211 \\
Date & 24686 & 99.252 & 99.481 & 98.673 & 98.901 \\
ID & 82 & 80.392 & 87.255 & 2.495 & 2.708 \\
Location & 16220 & 87.576 & 93.391 & 84.020 & 89.599 \\
Name & 9195 & 98.712 & 99.377 & 93.445 & 94.075 \\
Occupation & 157 & 60.153 & 68.966 & 62.800 & 72.000 \\
\end{longtable}

\paragraph{Goldstandard Part II}\label{goldstandard-part-ii}

\begin{longtable}[]{@{}llllll@{}}
\toprule\noalign{}
Type & Matches & Recall & Recall Lenient & Precision & Precision
Lenient \\
\midrule\noalign{}
\endhead
\bottomrule\noalign{}
\endlastfoot
Age & 355 & 92.689 & 92.689 & 96.995 & 96.995 \\
Contact & 2421 & 98.937 & 99.387 & 99.343 & 99.795 \\
Date & 22537 & 99.029 & 99.306 & 98.466 & 98.742 \\
ID & 105 & 79.545 & 84.848 & 4.009 & 4.276 \\
Location & 15207 & 82.999 & 89.417 & 84.620 & 91.164 \\
Name & 8227 & 98.870 & 99.603 & 95.841 & 96.552 \\
Occupation & 133 & 50.379 & 58.333 & 57.576 & 66.667 \\
\end{longtable}

\subsubsection{Annotation Issues
Observed}\label{annotation-issues-observed}

\paragraph{Age}\label{age}

Issues with more complex sentence structures, such as

\begin{itemize}
\tightlist
\item
  \texttt{Brüder\ verstorben\ mit\ 77,\ 71\ und\ 78\ Jahren}
\item
  \texttt{Sie\ sei\ eigentlich\ 49\ und\ nicht\ 42\ Jahre\ alt}
\end{itemize}

\paragraph{Contact}\label{contact}

Some hospital internal phone numbers were not recognized.

\paragraph{Date}\label{date}

Some scores are erroneously recognized as dates.

\begin{itemize}
\tightlist
\item
  \texttt{Nutritional\ Assessment:\ 7/15,\ Faszikulationen\ an\ 10/10\ Stellen}
\item
  \texttt{Beginn:\ 1.2,\ Albuminquotient\ 13.4}
\end{itemize}

\paragraph{ID}\label{id}

The extremely low precision comes from the fact, that what is considered
as IDs was changed. That is, some entities now annotated by the pipeline
are not annotated in the goldstandard. For the recall, some model
numbers were erroneously annotated in the gold standard

\paragraph{Location}\label{location}

The somewhat medium performance in the location category originates in
the broad definition including place names and terms referring to names
of organisations or organisational units. It is also not always clear
whether terms like \texttt{Physiotherapie}, \texttt{Unfallchirurgie},
\texttt{Innere\ Medizin} and the like are part of an organization name
or are just generic medical terms not carrying any identifying meaning.
This sort of ambiguous case make up the largest part.

A more detailed look reveals, that the pipeline missed very few patient
related location information, that is a 7 locations outside Switzerland
in Goldstandard Part I. About 10 organisations were missed.

\paragraph{Name}\label{name}

Great care was taken to not miss names of patients or staff, i.e.~the
pipeline is tuned for a high recall. The comparatively low precision is
due to the fact, that some staff abbreviations were not annotated in the
goldstandard corpus. Another problem are that medical terms like
\texttt{M.\ Scheuermann}, \texttt{Spina}, \texttt{Carina},
\texttt{Karina}, \texttt{B.\ Fieber} get annotated as names.

\paragraph{Occupation}\label{occupation}

This is a difficult category to annotate, since the way professions or
occupations can be expressed are quite variable. They typically occur
in certain fields related to anamnesis. These can be excluded using
the \texttt{-\/-fields-blacklist} option in the substitute command
(Section~\ref{substitution-policies}) if the risk is too high.
