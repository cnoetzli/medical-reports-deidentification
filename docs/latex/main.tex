\documentclass[10pt,a4]{article}
\usepackage{fullpage}
\usepackage{graphicx}
\usepackage{hyperref}
\usepackage{longtable}
\usepackage{array}
\usepackage{booktabs} % Required for using \toprule, \midrule, and \bottomrule in tables

\usepackage{calc}  % Add this to handle calculations
\newcommand{\real}[1]{#1}  % Define real if not already defined

\def\tightlist{%
  \setlength{\itemsep}{0pt}\setlength{\parskip}{0pt}}

\title{De-Identification Tool for Clinical Reports}
\author{Marc Zimmermann$^1$, Katie Kalt$^2$, Patrick Hirschi$^2$, and\\
  Gunnar R\"atsch$^{1,2}$\\
{\footnotesize $^1$ Biomedical Informatics Group, Department of Computer Science, ETH Zurich, Zurich, Switzerland}\\
{\footnotesize $^2$ Department Research and Teaching, University Hospital Zurich, Zurich, Switzerland}}
\date{October 2021}

\begin{document}

\maketitle

\begin{abstract}
This document outlines the development and implementation of a robust
software tool designed to de-identify clinical reports to address
privacy regulations and policies. The tool employs advanced natural
language processing techniques to accurately identify and anonymize
personal health information (PHI) from clinical texts, making them
suitable for further research and analysis without compromising
patient confidentiality. Key components of the software include a
customizable annotation pipeline, extensive lexica for precise entity
recognition, and sophisticated algorithms for sensitive data detection
and substitution. The document also details the software's
architecture, setup requirements, usage guidelines, and performance
metrics, supported by a case study on its application in a real-world
healthcare setting. This comprehensive approach not only supports
using the tool to meet regulatory requirements but also adapts
efficiently to varied data formats and clinical environments.
\end{abstract}

\newpage

\tableofcontents
\newpage

\section{Introduction}
%\section{Overview}\label{overview}

The deidentification tool consists of a collection of command line
applications written in Java. The applications is based on the
\href{https://gate.ac.uk/}{GATE framework}

Here, a short overview over each command line application. Each square
box represents a different command described in the following sections.

\begin{figure}
\centering
\includegraphics[width=\textwidth]{figs/system_overview.png}
\caption{System overview}
\end{figure}

\subsection{Document Import}\label{document-import}

The deidentifier tool imports JSON reports typically from a database
(json reports on the filesystem are also supported) and converts them to
a GATE compatible representation. The \texttt{annotate} command can read
directly from the appropriate source. The \texttt{import} command only
does the import and conversion step and stores a batch of documents into
a GATE corpus (which is a directory on a filesystem).

It is assumed that the documents are stored in a database table or view,
one row per document. The actual report content is encoded as JSON
string in one of the columns. Other required columns denote the document
type (\texttt{FCODE}) as well as the report id.

The tree like structure of the JSON documents is preserved during the
conversion to the GATE compatible representation. This can be exploited
during the annotation.

\subsection{annotate}\label{annotate}

The \texttt{annotate} command takes documents from a GATE corpus and
runs an annotation pipeline over the reports, i.e.~annotates portions of
the text which contain entities to be deidentified. The output of this
process is again a GATE corpus, which can be examined e.g.~using the
\href{https://gate.ac.uk/download/}{GATE developer tool}.

An \emph{annotation} simply denotes a span of text with some properties
associated, for example: ``Mr.~Muster, born 01.01.1964 in Aarau'' could
have 3 annotations related to deidentification: one for `Muster' (Name),
another for `01.01.1964' (Date, with the additional information that it
is a birthdate) and `Aarau' (Location).

Currently, the following entities are annotated: * Age * Contact
(distinguishing phone numbers, email and websites) * Date (if possible
determining birth date, admission date, discharge date) * ID (patient or
case ID, social security or insurance numbers) * Name (if possible
distinguishing patient from staff) * Location (broad category containing
geographical locations as well as organizations) * Occupation

The annotation pipeline consists basically of the following consecutive
steps:

\begin{figure}
\centering
\includegraphics[width=0.2\textwidth]{figs/pipeline_overview.png}
\caption{Pipeline steps}
\end{figure}

\begin{enumerate}
\def\labelenumi{\arabic{enumi}.}
\tightlist
\item
  \textbf{Tokenization}: splitting the text into units of characters
  (`words'), for example `Mr.~Muster' would be split into 3 tokens `Mr',
  `.' and `Muster'
\item
  \textbf{Sentence Splitting}: Grouping tokens together into sentences
\item
  \textbf{Lookups}: Annotating tokens using dictionaries/lexica (called
  ``Gazetteers'' in GATE). For instance, the tokens `Universitätsspital
  Zürich' could be annotated as ``organsation'' and the token ``Zürich''
  as ``location''
\item
  \textbf{Field Normalization}: Renaming some fields in the report to a
  common name to simplify downstream steps. E.g. \texttt{AddrLine},
  \texttt{Address}, \texttt{Adresse} could all be renamed to
  \texttt{AddressField}.
\item
  \textbf{Structured Annotation}: annotate entire fields known to
  contain information to be deidentified. For example, the content of a
  field \texttt{Tel} denoting a phone number could immediately be
  annotated without having to apply any further processing.
\item
  \textbf{Context Annotation}: annotate a window (of tokens) around some
  trigger token with a \href{components.md\#context-annotations}{context
  annotation}. like \texttt{NameContext} or \texttt{OccupationContext}
  to help JAPE rules to disambiguate between e.g.~surnames and
  professions.
\item
  \textbf{JAPE rules}: annotate tokens based on a regular
  expression-like language. This is based on the structure or content of
  tokens (e.g.~a specific trigger word such as ``Dr'' or being a number)
  as well as previous annotations from dictionaries from the previous
  step. Note, that rules can also exploit the tree-like structure of the
  document, for example a rule may only apply if the token in a field is
  part of the section of the document related to patient information.
\item
  \textbf{High Confidence String Annotator}: Some JAPE rules in the
  previous step can be marked as \texttt{high\ confidence}, that is,
  whatever these rules annotate, we have high confidence that it is
  correct. The \texttt{HighConfidenceStringAnnotator} then checks what
  tokens were annotated by a high confidence rule and then annotates the
  same tokens in the remaining document.
\item
  \textbf{Clean up}: resolve overlapping or conflicting annotations
  using heuristics.
\end{enumerate}

\subsubsection{JAPE Example}\label{jape-example}

Here an example of a JAPE rule. We are interested in recognizing ages in
a very specific pattern, namely the age followed by ``jährige'',
``jähriger'' or ``jährigen'', e.g.~``59-jähriger Patient'':

\begin{verbatim}
// 0-119 (including decimals)
Macro: POSSIBLE_AGE
(
    ({Token.string ==~ "[1-9]*[0-9]"} | {Token.string ==~ "1[0-1][0-9]"})
    ({Token.string == "."} {Token.string ==~ "[0-9]+"})?
)


Rule: AgeRightContextTrigger
(
   (POSSIBLE_AGE):age
   ({Token.string == "-"})?
   ({Token.string ==~ "jährige[rn]?"})
)
-->
:age.Age = {rule = "AgeRightContextTrigger"}
\end{verbatim}

Rules describe a sequence of tokens on the ``left hand side'',
i.e.~before ``--\textgreater{}''. If such a sequence is recognized in a
text a rule is triggered and the ``right hand side'' is applied. In the
above case, the right hand side adds an annotation of type \texttt{Age}
having as a property the name of the rule triggered (this helps
debugging).

A token can be described exactly, like
\texttt{\{Token.string\ ==\ "-"\}} where the token should consist
exactly of \texttt{-} or using regular expressions as in
\texttt{\{Token.string\ ==\textasciitilde{}\ "1{[}0-1{]}{[}0-9{]}"\}}
describing the numbers from 100 to 119. A sequence of tokens may contain
optional elements denoted with \texttt{?}. In the above example the
\texttt{(\{Token.string\ ==\ "-"\})?} signifies, that there may or may
not be a dash.

More details regarding JAPE can be found in the
\href{https://gate.ac.uk/sale/thakker-jape-tutorial/GATE\%20JAPE\%20manual.pdf}{JAPE
Grammar Tutorial}

Note, that the above example is not robust against typos, e.g.~the rule
would fail to annotate the age in ``59-järiger Patient''.

\subsection{substitute}\label{substitute}

Takes an annotated GATE corpus and generates a JSON representation of
the content with to be deidentified tokens replaced. The JSON version of
these reports can then be saved back to a database table or to JSON
files on a drive one per report.

\subsubsection{Substitution Policies}\label{substitution-policies}

There are several policies implemented on how annotated tokens should be
replaced:

\begin{itemize}
\tightlist
\item
  \texttt{ScrubberSubstitution}: entities are replaced by a fixed string
  depending on the annotation type, for example `am 01.02.2003' would be
  replaced as ``am DATE'' and `Dr.~Muster' by `Dr.~NAME'
\item
  \texttt{DateShift}: same as \texttt{ScrubberSubstitution}, but all
  dates of a report are shifted by a random amount of days into the
  future or past.
\item
  \texttt{ReplacementTags}: In this policy information are passed along
  to a downstream application which takes care of the actual
  deidentification. For that purpose entities are replaced by `tags'
  which contain as much information as possible from the annotation
  pipeline. For example a text like ``Dr.~P. Muster empfiehlt'' could be
  replaced by
  \texttt{Dr.\ {[}{[}{[}Name;P.\ Muster;firstname=P;lastname=Muster;type=medical\ staff{]}{]}{]}\ empfiehlt}
  that is, the original value is preserved and the downstream
  application can decide how to replace the name most appropriately.
\end{itemize}

There exist also the \texttt{-\/-fields-blacklist} option, where a list
of field names can be provided which are completely erased from the
document. This can be useful for fields with are notoriously hard to
deidentify, but contain no relevant information for a downstream
application.


\section{System Setup and Requirements}
\input{installation.tex}

\section{Architectural Overview}
\input{code_overview.tex}

\section{Key Components and Configurations}
\subsection{Annotation Pipeline Components and theirs
Configurations}\label{annotation-pipeline-components-and-theirs-configurations}

The deidentification tool was designed to be relatively flexible to
accommodate various needs.

\subsection{Data Input}\label{data-input}

More details on the configuration of the report input data source.

\subsubsection{DB Configuration}\label{db-configuration}

Configuration parameters to load reports from a database are stored in a
text file (properties file) with the following attributes
(example config in Section~\ref{adapt-database-configuration}):

\begin{itemize}
\tightlist
\item
  \texttt{jdbc\_url}: the URL to connect to the database,
  \href{https://docs.oracle.com/javase/tutorial/jdbc/basics/connecting.html}{see
  also}
\item
  \texttt{user}: the database user name
\item
  \texttt{password}: the password
\item
  \texttt{query}: a ``SELECT'' SQL query. Can be anything (e.g.~joining
  data from several tables, a view \ldots) as long as certain columns
  are present
\item
  \texttt{json\_field\_name}: the column name in the above SQL query
  denoting the JSON content of a report
\item
  \texttt{reportid\_field\_name}: the column name denoting the reportid
  of a report in the SQL query
\item
  \texttt{report\_type\_id}: the column name in the above query denoting
  the report type (``FCODE'') of a report. This is mainly used to select
  certain types of reports.
\item
  \texttt{date\_field\_name}: the column name in the above query
  denoting the creation date of a report (optional).
\end{itemize}

In case reports should be written back to a database after substitution:
* \texttt{dest\_table}: table name to write into *
\texttt{dest\_columns}: the names of the columns to write back. These
should be a subset of the columns in the above SELECT SQL query (could
also be all of them)

\paragraph{Report Filtering}\label{report-filtering}

\subparagraph{Document Type Filter}\label{document-type-filter}

Depending on the project, only certain document types (``FCODE'') might
be relevant. These could be filtered out in the SQL query or also using
a simple text file which can be passed to the \texttt{annotate} or
\texttt{import} command using the \texttt{-\/-doc-type-filter} option.

The file contains one row per document type and at least one column
(seperated by `,'), where the first column denotes the document type
name. There can be more columns (for example human readable
description), which are ignored by the application.

\subparagraph{Document ID Filter}\label{document-id-filter}

Similar to document type filters, one can specify to load only documents
having a specific ID. This can be done by passing a file path using the
\texttt{-\/-doc-id-filter} option.

Columns: * report id

\subsection{Annotation Pipeline}\label{annotation-pipeline}

Quite a few aspects of the annotation pipeline can be parameterized. In
this section, more details about various annotation steps and their
configuration.

\subsubsection{Pipeline Configuration
File}\label{pipeline-configuration-file}

Many pipeline steps can be parametrized by specific configuration files
or other parameters. The parametrization happens via a configuration
file setting all relevant parameters for the annotation pipeline. You
can pass the path of the file to the \texttt{annotate} command using
\texttt{-c}. The syntax of the file follows the
\href{https://github.com/lightbend/config}{HOCON format}, see the
\href{https://github.com/ratschlab/medical-reports-deidentification/blob/main/configs/kisim-usz/kisim_usz.conf}{configuration of the USZ
pipeline as example}

Configurations relevant to the pipeline are grouped together into the
\texttt{pipeline} `section'.

\subsubsection{Lexica (Dictionaries,
Gazetteers)}\label{lexica-dictionaries-gazetteers}

The \texttt{pipeline.gazetteer} option should point to a GATE gazetteer
file definition (\texttt{*.def}). This text file contains an entry for
each dictionary file with the (relative) path and the annotation type. A
dictionary file is simply a text file with one or more token per line.
More details in the
\href{https://gate.ac.uk/sale/tao/splitch6.html\#x9-1270006.3}{GATE
Documentation}

The annotation pipeline also uses a second category of gazetteers
specified in \texttt{pipeline.suffixGazeteer} not matching entire tokens
but suffixes. This is useful for rules based on word endings, for
example to recognize surnames (``-mann'', ``-oulos'', ``-elli'') and
medical terms (``-karzinom'', ``-suffizienz'', ``-beschwerden'').

\subsubsection{Specific JAPE Rules}\label{specific-jape-rules}

There is a generic set of JAPE rules shipped with the application.
Typically, these rules cannot cover special cases appearing in a given
organization. This can be done using a separate rule set.

Specific rules can be added via the \texttt{pipeline.specificTransducer}
option pointing to a \texttt{*.jape} file. This file would contain a
list of different phases, where every phase is a separate
\texttt{*.jape} file. These files would then contain the actual JAPE
rules. See also the
\href{https://gate.ac.uk/sale/tao/splitch8.html\#x12-2310008.5}{Gate
Documentation}.

\subsubsection{Report Structure}\label{report-structure}

If available, the structure of input documents can be exploited during
the annotation (and in principle also during substitution). That is,
prior knowledge about the document can be added.

In case of JSON, the structure elements would be field names and nested
objects. A JSON document can then be seen as a tree where the leaves
contain the actual report text fragments.

\begin{figure}
\centering
\includegraphics[width=\textwidth]{figs/report_structure.png}
\caption{simple report structure}
\end{figure}

\paragraph{Paths in Field Tree}\label{paths-in-field-tree}

In the various configuration files related to annotations, paths in the
``field tree'' can be used to denote certain parts of the document
(similar to XPath for XML documents).

A \textbf{path} can consist of the following elements: * field name: can
be a
\href{https://docs.oracle.com/javase/8/docs/api/java/util/regex/Pattern.html}{regular
expression accepted by Java} * `/' denoting that the nodes must appear
consecutively, e.g.~\texttt{field\_a/field\_c} matches only
``field\_a/field\_c'' but not ``field\_a/field\_b/field\_c'' . * `//'
denoting that the nodes don't need to be necessarily consecutive, e.g.
\texttt{field\_a/field\_c} would match both ``field\_a/field\_c'' and
``field\_a/field\_b/field\_c''

Note, that paths are case-sensitive.

Examples:
Simple field names which can appear anywhere in the tree: *
\texttt{Id} * \texttt{ID} * some regular expression
\texttt{{[}\textbackslash{}\textbackslash{}p\{IsAlphabetic\}{]}*VisDat}
(``\\p'' is used to match unicode characters). * Field names with some
constraints regarding the parent. For instance, to match \texttt{Text}
fields only when it is a child of \texttt{PatInfo}, you can use
\texttt{//PatInfo//Text}, which would match
e.g.~``/report/PatInfo/some\_element/Text'' * Field constraints
``anchored'' from the top: \texttt{/NOTE} would match a field ``NOTE''
directly under the root of a tree, but not ``/PatientInfo/NOTE''.

\paragraph{Structured Annotations}\label{structured-annotations}

The ``structured annotation step'' allows annotating entire text fields.
For instance, if you know that a certain field contains the name of a
person, then an annotation can be performed at that level.

This step can be parametrized with a configuration file passed in
\texttt{pipeline.structuredFieldMapping} in the pipeline configuration.

Columns in the config file (separated by ``;''): * Path * Annotation
type (e.g.~\texttt{Name}, \texttt{Date} etc) * Features (properties) of
the annotation . Features are separated by ``,'' where key and values
are separated by ``=''

Example: \texttt{//Personalien//Name;\ Name;\ type=patient}

All leaves with \texttt{Name} having \texttt{Personalien} as a parent
somewhere are annotated with \texttt{Name} and having the \texttt{type}
property set to \texttt{patient}.

\paragraph{Field Normalization (Field Annotation
Mapping)}\label{field-normalization-field-annotation-mapping}

Sometimes, we can not blindly annotate an entire field, but need to
apply a JAPE rule on it. For example, signatures could have a structure
like ``ABCD, 20.10.2018'' where ``ABCD'' is the shorthand for a doctor.
Since there are many fields with similar or identical structure, but
different paths the fields can get renamed to a common name. A JAPE rule
processing the pattern would then refer to that common name.

This step can be parametrized with a configuration file passed in
\texttt{pipeline.annotationMapping} in the pipeline configuration.
Columns (separated by ``;''): * Path * New field name

Example: \texttt{//Patient/Alter/Val;\ AgeField}

A field \texttt{Val} with immediate ancestors \texttt{Alter} and
\texttt{Patient} gets named an \texttt{AgeField}. Now a JAPE rule only
working on \texttt{AgeField}s, could for example annotate any number in
there as an age.

\paragraph{Annotation Blacklist}\label{annotation-blacklist}

For some fields we can exclude a priori certain annotations. An example
could be a field containing computer generated identifiers like
\texttt{9b02d92c-c16e-4d71-2019-280237bb8cb5} where a JAPE rule may
erroneously pick up a date (for example ``2019'' in the example). A
blacklisting step would remove such annotations.

This step can be parametrized with a configuration file passed in
\texttt{pipeline.annotationBlacklist} in the pipeline configuration.

Columns (seperated by ``;''): * Path * Comma separated annotation types
which should \emph{not} appear within elements denoted in path

Example: \texttt{//DiagnList//CodeList//Version;\ Date}

The \texttt{Version} field having \texttt{CodeList} and
\texttt{DiagnList} as parents should not contain \texttt{Date}
annotations.

\subsubsection{Context Annotations}\label{context-annotations}

There are some JAPE rules which only get triggered if tokens appear in a
specific language context. This can be useful to disambiguate between
e.g.~surnames and professions or between the profession of a patient vs
the role of staff.

The context can be given by a field (using
\hyperref[field-normalization-field-annotation-mapping]{Annotation Mappings}) or by using
\textbf{context annotations}. They can be added around trigger tokens.
For instance, text in the vicinity of \texttt{Sohn} (son) may contain
his name or information about his profession. Therefore,
\texttt{NameContext} and \texttt{OccupationContext} context annotations
are added to the document, spanning the e.g.~5 tokens before
\texttt{Sohn} and 5 tokens after. Later on, if within these context
annotations e.g.~an isolated first name appears, it can be annotated as
\texttt{Name} since we assume it is a context where names can occur
(otherwise we wouldn't annotate it, as there is not enough evidence).
Context annotations are performed in early stages of the pipeline, s.t.
they can be referred to in JAPE rules later on.

These context triggers can be configured in a config file whose path has
to be provided in \texttt{pipeline.contextTriggerConfig} in the pipeline
configuration.

Columns (separated by ``;''): * Context token (no spaces) * Name of the
context (e.g.~\texttt{NameContext}) * start of the context annotation in
number of tokens before the trigger token * end of the context
annotation in number of tokens after the trigger token

Examples: * \texttt{Sohn;NameContext;5;5} *
\texttt{Partner;OccupationContext;5;5}

\subsection{Test Suite}\label{test-suite}

A small testing framework was developed to test the annotation behavior
of the pipeline in a fast and isolated way. That is, small test cases
can be defined consisting of a phrase and the annotations the pipeline
is expected to produce. This allows for test driven development/tuning
of the annotation pipeline.

\subsubsection{Test Cases Specification}\label{test-cases-specification}

The testcases are described in a textfile. The first line of the text
file contains the annotation types the pipeline is tested against as
well as the context fields. The context fields annotations are used to
test rules based on the document structure. The lists for annotation
types and context fields are seperated by \texttt{;} and the entries in
lists by \texttt{,}.

Then testcases follow, one per line. Manual annotations and fields are
added using XML-tags. Comments using `\#' are allowed either to comment
entire lines are the remainder of a line. Commented parts are ignored.
There may be empty lines for making the file a bit more readable. If new
lines are needed to test a specific situation, this can simply be done
using \texttt{\textbackslash{}n}.

Following an example with 3 test cases:

\begin{verbatim}
Name; FieldWithSignature

Der Patient <Name>Luigi D'Ambrosio</Name> wurde...

<FieldWithSignature>20.01.2018 / <Name>AMUSTER</Name> </FieldWithSignature>
20.01.2018 / AMUSTER # don't expect name annotation in arbitrary fields
\end{verbatim}

In this example, the annotation of \texttt{Name} is tested. The
\texttt{\textless{}Name\textgreater{}} tags are removed before the test
case is passed through the pipeline. Then, the \texttt{Name} annotations
of the pipeline output are checked whether they indeed contain
\texttt{Name} annotations at the same place, and only there. If this is
the case, the test passes, otherwise it fails with an appropriate
message.

The second test case tests a context specific rule, i.e.~the rule is
only applied within fields \texttt{FieldWithSignature} (In the USZ
pipeline, \texttt{FieldWithSignature} annotation is added the annotation
mapping step, see above) The third test case is just to see, if the
previous rule is not triggered outside the required context. Or said
differently, we expect \texttt{AMUSTER} not to be annotated,
i.e.~annotating it would be wrong.

In some existing tests there is also the \texttt{OUTOFVOC} token. It
stands for ``out of vocabulary'' and makes it explicit, that the rule
should rely exclusively on structure, and not be based on entries in the
dictionary.

\subparagraph{Running Tests}\label{running-tests}

A test suite can be run using the \texttt{test} command from the
\texttt{DeidMain} entry point:

\begin{verbatim}
$DEID_CMD test [pipeline configuration file] [testcase directory]
\end{verbatim}

It commands needs a path to a pipeline configuration file
(e.g.~\texttt{configs/kisim-usz/kisim\_usz.conf}) and a directory with
testcases (e.g.~\texttt{configs/kisim-usz/testcases/}). Every
\texttt{*.txt} in that directory is assumed to contain test cases.

The generic rules shipped with the tool are tested using the same
mechanism. They are run as unit tests for the tool itself. The test
cases can be found in the directory
\texttt{deidentifier-pipeline/src/test/resources/pipeline\_testcases}
You normally don't need to modify these tests while tuning the pipeline,
but you may consider them a source of useful examples.


\section{Dictionary and Lexicon Assets}
\input{lexica.tex}

\section{Software Development and Updates}
\input{development.tex}

\section{Data Processing and Workflow Structuring}
\input{structuring.tex}

\section{Annotation Rules and Optimization Strategies}
\input{rules_and_tuning.tex}

\section{Guidelines for System Tuning}
\subsection{Tuning Tutorial}\label{tuning-tutorial}

Here few hands-on ``exercises'' on how to tune the pipeline on a heavily
simplified version of the USZ deidentification pipeline. The goal is to
get familiar with the various components in a simplified setting without
being overwhelmed by all the details of a full-fledged pipeline.

\subsubsection{Setup}\label{setup}

To test whether modification of the pipeline lead to the desired
behavior, we are going to use the testing framework included in the
deidentification tool. Tests for many of the exercises below are already
prepared in the \texttt{configs/tutorial/testcases}, they just need to
be uncommented (i.e.~removing the \texttt{\#})

The test suite can be run using the following command:

\begin{verbatim}
java -jar [path to jar file] test [pipeline config file] [test cases directory]
\end{verbatim}

where \texttt{{[}path\ to\ jar\ file{]}} should point to a current
\texttt{jar} file of the pipeline (typically
\texttt{deidentifier-*.jar}), \texttt{{[}pipeline\ config\ file{]}} to
some path ending with \texttt{configs/tutorial/tutorial.conf} and
\texttt{{[}test\ cases\ directory{]}} a path ending on
\texttt{configs/tutorial/testcases}. Tip: put the resulting long command
into a \texttt{.bat} or \texttt{.sh} file which you then execute.

When running the test suite, if everything goes well, you should see
lines containing \texttt{Reading\ testcases\ from} at the end. If a
testcase should fail, a clear error message is displayed with some more
details what went wrong.

\subsubsection{Exercices}\label{exercices}

\paragraph{Add test case for dates}\label{add-test-case-for-dates}

In the file \texttt{date.txt} there is already one test case defined to
check whether a date is indeed recognized. Uncomment that line and run
the test suite. You should see something like

\begin{verbatim}
2019-12-06 13:27:40.891 INFO  org.ratschlab.deidentifier.pipelines.testing.PipelineTestSuite - Reading testcases from configs/tutorial/testcases/date.txt
2019-12-06 13:27:40.895 INFO  org.ratschlab.deidentifier.pipelines.testing.PipelineTester - Running test suite with 1 test cases
\end{verbatim}

On a new line add another date without the
\texttt{\textless{}Date\textgreater{}} tags. Run the test suite again
and see how it fails. Add the tags, run again and this time the suite
should pass.

\paragraph{Add missing location}\label{add-missing-location}

The location \texttt{Oberikon} is not recognized in a document. Add a
test case for it in the \texttt{locations.txt} file and run the test
suite. It should fail on that test. Then, add the place to some
appropriate lexikon (e.g.~in the already existing
\texttt{locations/additional\_locations.lst}). After that, the test
suite

\paragraph{Internal phone number
format}\label{internal-phone-number-format}

Assume internal phone numbers consist of two blocks of 3 digits, e.g.:
\texttt{123\ 456}. Write a JAPE rule which recognizes these numbers.
There is already some test case in \texttt{contact.txt}

Hints: * add the rule in \texttt{specific-rules/contacts.jape}. There is
already some rule recognizing some Swiss phone numbers (copied from the
generic JAPE rule set). * See
https://gate.ac.uk/sale/thakker-jape-tutorial/GATE\%20JAPE\%20manual.pdf
if you'd like to know more about how JAPE rules work. * In practice, you
would probably add a ``trigger'' on the left side, i.e.~fire the rule
only if the two blocks are preceded by a ``Tel'' token.


\section{Performance Evaluation and Metrics}
\input{evaluation.tex}

\section{Case Study: Pipeline Performance on Report from University Hospital Zurich}
\input{usz_pipeline_evaluation.tex}

\section{Discussion and Summary}
%\section{Discussion and Summary}
This document has detailed the development and deployment of a
sophisticated de-identification tool for clinical reports, designed to
ensure privacy and compliance with health data regulations. Through
rigorous testing, including a focused case study at the University
Hospital Zurich, the tool has demonstrated high effectiveness in
recognizing and anonymizing personal health information
(PHI).

{\bf Notably, the tool achieves over 99\% recall in identifying and
  anonymizing 'Contact', 'Date', and 'Name' entities, and
  approximately 95\% for 'Age', with 'Location' entities recognized
  with about 91\% recall.} These results underscore the tool's robust
capability to safeguard sensitive information.  Challenges remain in
the lower recall rates for 'ID' and 'Occupation', which are 85\% and
64\%, respectively. However, these entities are considered less
critical from a privacy standpoint as 'ID' numbers appear infrequently
and occupations are generally not too specific. {\bf If a recall
  threshold of at least 90\% is set as a benchmark for privacy
  concerns, this tool reliably removes critical PHI categories such as
  Age, Contact, Date, Location, and Names.}

Future improvements will focus on enhancing the tool's machine
learning models to better understand contextual nuances and reduce
false positives, thereby increasing the reliability of the
de-identification process. This ongoing enhancement will ensure that
the tool not only meets the regulatory requirements but also adapts
efficiently to varied data formats and clinical environments,
maintaining high standards of patient privacy without compromising the
utility of the data for research and clinical review.



\end{document}
