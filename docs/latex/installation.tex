\subsection{Installation}\label{installation}

\subsubsection{Prerequisites}\label{prerequisites}

\begin{itemize}
\tightlist
\item
  at least Java 8
\item
  appropriate JDBC driver, if reports are loaded from database
  (PostgreSQL drivers are already included)
\end{itemize}

\subsubsection{Getting the Software}\label{getting-the-software}

Download the latest release from the
\href{https://github.com/ratschlab/medical-reports-deidentification/releases}{releases
page}, that is, the jar file in the \texttt{Assets} section.

Alternatively (more advanced), you can download a recent zip archive
generated everytime the github action workflows are triggered. You can
look for
\href{https://github.com/ratschlab/medical-reports-deidentification/actions}{workflow
runs of the branch ``main''} and look for the ``Artifacts'' section of a
specific run.

\paragraph{Building Yourself}\label{building-yourself}

You need'll need \href{https://maven.apache.org/install.html}{Maven} to
build from source code. Once \texttt{maven} is set up, you can run the
following in the root directory of the repository

\begin{verbatim}
mvn package --file deidentifier-pipeline/pom.xml
\end{verbatim}

You'll then find the \texttt{jar} file containing the pipeline code as
well as its dependencies in \texttt{deidentifier-pipeline/target}

\subsubsection{Basic Usage Example}\label{basic-usage-example}

In the following, a small example to deidentify a few JSON files.

First, create a directory \texttt{orig\_reports} and populate it with
\href{https://github.com/ratschlab/medical-reports-deidentification/blob/main/deidentifier-pipeline/src/test/resources/kisim_simple_example.json}{an
example file} and/or create your own JSON files to put there.

The basic invocation of the deidentification tool is

\begin{verbatim}
java -jar [path to deidentifier-VERSION.jar]
\end{verbatim}

In the following we abbreviate this by \texttt{DEID\_CMD} (on a shell
you could e.g.~run
\texttt{DEID\_CMD="java\ -jar\ deidentifier-1.0.jar"}).

To annotate terms which need to be deidentified in the documents, run

\begin{verbatim}
$DEID_CMD annotate -i orig_reports --json-input -o annotated_reports -c configs/kisim-usz/kisim_usz.conf
\end{verbatim}

You may have to adapt the path to the
\href{https://github.com/ratschlab/medical-reports-deidentification/blob/main/configs/kisim-usz/kisim_usz.conf}{kisim\_usz.conf} file. The
output in \texttt{annotated\_reports} can be opened and inspected using
\href{https://gate.ac.uk/download/}{GATE Developer}, the graphical user
interface of the GATE framework.

The annotations can now be replaced and written out to disk again.

\begin{verbatim}
$DEID_CMD substitute -o substituted_reports --method Scrubber annotated_reps
\end{verbatim}

The substituted reports can now be found in
\texttt{substituted\_reports}, where annotated terms are replaced by a
fixed string (more about this in the Section~\ref{substitution-policies}.

\paragraph{Adapt Database
Configuration}\label{adapt-database-configuration}

In case reports should be read from a database instead of a directory, a
configuration file needs to be created specifying host, username,
password table etc. You can find more details in Section~\ref{db-configuration}.

In the following we'll assume the file is called \texttt{db\_conf.txt}.

Here an example, how the file could look like:

\begin{verbatim}
jdbc_url=jdbc:sqlserver://myhost:2345;databaseName=MyDB;
user=deid_poc
password=1asdffea
query=SELECT DAT,FALLNR,CONTENT,FCODE,REPORTNR FROM MyDB.KISIM_KIS_T_REPORT_JSON.KIS_T_REPORT_JSON
json_field_name=CONTENT
reportid_field_name=REPORTNR
report_type_id_name=FCODE
date_field_name=DAT

# for writing back
dest_table=subst_test
dest_columns=CONTENT,REPORTNR,FCODE,DAT,FALLNR
\end{verbatim}

Reports are read from the \texttt{KISIM\_KIS\_T\_REPORT\_JSON} table of
the \texttt{MyDB} mssql database. Substituted reports are written back
into the table \texttt{subst\_test} into the column \texttt{CONTENT} and
the columns \texttt{REPORTNR,FCODE,DAT,FALLNR} are just copied over as
is.

For a postgres DB the JDBC URL would start with
\texttt{jdbc:postgresql://...}. Other databases are supported in
principle, but the corresponding JDBC driver needs to be made available
on the java classpath.

To annotate reports from a database table, the command would become:

\begin{verbatim}
$DEID_CMD annotate -d db_conf.txt -o annotated_reports -c configs/kisim-usz/kisim_usz.conf
\end{verbatim}

The \texttt{annotate} command provides some basic mean to select
appropriate reports via the options \texttt{-\/-max-docs},
\texttt{-\/-skip-docs}, \texttt{-\/-doc-id-filter} and
\texttt{-\/-doc-type-filter} (see
\texttt{\$DEID\_CMD\ annotate\ -\/-help} for more details) More complex
filtering could be done by tweaking the \texttt{query} field in
\texttt{db\_conf.txt}.

To substitute the annotated reports and write them back into another
database table, the command would be:

\begin{verbatim}
$DEID_CMD substitute -d db_conf.txt --method Scrubber annotated_reps
\end{verbatim}

\subsubsection{Further Tips for Running the Deidentifier
Tool}\label{further-tips-for-running-the-deidentifier-tool}

\paragraph{File Encoding}\label{file-encoding}

If you run into encoding related issues, try adding
\texttt{-Dfile.encoding=UTF-8} into your \texttt{DEID\_CMD}, e.g

\begin{verbatim}
java -Dfile.encoding=UTF-8 -jar deidentifier-*.jar
\end{verbatim}

\paragraph{Increasing Memory}\label{increasing-memory}

If the tool crashes with e.g.~an \texttt{OutOfMemoryError} or the
processing is very slow, try increasing the memory the Java virtual
machine (jvm) is allowed to use using the \texttt{-Xmx} option, e.g.

\begin{verbatim}
java -Xmx4g -jar deidentifier-*.jar
\end{verbatim}

Here, in total at most 4g of RAM would be used.

\paragraph{Customize Logging}\label{customize-logging}

We use \texttt{log4j2} to manage logs. You can provide a custom
\texttt{log4j} config by passing
\texttt{-Dlog4j.configurationFile={[}path\ to\ log\ config{]}} to the
java command. The default logging configuration can be found
\href{https://github.com/ratschlab/medical-reports-deidentification/blob/main/deidentifier-pipeline/src/main/resources/log4j2.xml}{here}. See
the
\href{https://logging.apache.org/log4j/2.x/manual/configuration.html}{log4j
documentation} for more infos.

To debug the logging setup, you can add the \texttt{-Dlog4j.debug} flag
to the java command.

\paragraph{Adding JDBC Drivers}\label{adding-jdbc-drivers}

To connect to a database other than Postgres, you need to download an
appropriate JDBC driver (make sure it is compatible with both the java
and database version you are using).

The command line invocation (the \texttt{DEID\_CMD}) becomes then

\begin{verbatim}
java -cp "deidentifier-pipeline/target/deidentifier-0.99.jar;[path to jdbc jar]" org.ratschlab.deidentifier.DeidMain
\end{verbatim}
