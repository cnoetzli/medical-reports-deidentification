\subsection{Tuning Tutorial}\label{tuning-tutorial}

Here few hands-on ``exercises'' on how to tune the pipeline on a heavily
simplified version of the USZ deidentification pipeline. The goal is to
get familiar with the various components in a simplified setting without
being overwhelmed by all the details of a full-fledged pipeline.

\subsubsection{Setup}\label{setup}

To test whether modification of the pipeline lead to the desired
behavior, we are going to use the testing framework included in the
deidentification tool. Tests for many of the exercises below are already
prepared in the \texttt{configs/tutorial/testcases}, they just need to
be uncommented (i.e.~removing the \texttt{\#})

The test suite can be run using the following command:

\begin{verbatim}
java -jar [path to jar file] test [pipeline config file] [test cases directory]
\end{verbatim}

where \texttt{{[}path\ to\ jar\ file{]}} should point to a current
\texttt{jar} file of the pipeline (typically
\texttt{deidentifier-*.jar}), \texttt{{[}pipeline\ config\ file{]}} to
some path ending with \texttt{configs/tutorial/tutorial.conf} and
\texttt{{[}test\ cases\ directory{]}} a path ending on
\texttt{configs/tutorial/testcases}. Tip: put the resulting long command
into a \texttt{.bat} or \texttt{.sh} file which you then execute.

When running the test suite, if everything goes well, you should see
lines containing \texttt{Reading\ testcases\ from} at the end. If a
testcase should fail, a clear error message is displayed with some more
details what went wrong.

\subsubsection{Exercices}\label{exercices}

\paragraph{Add test case for dates}\label{add-test-case-for-dates}

In the file \texttt{date.txt} there is already one test case defined to
check whether a date is indeed recognized. Uncomment that line and run
the test suite. You should see something like

\begin{verbatim}
2019-12-06 13:27:40.891 INFO  org.ratschlab.deidentifier.pipelines.testing.PipelineTestSuite - Reading testcases from configs/tutorial/testcases/date.txt
2019-12-06 13:27:40.895 INFO  org.ratschlab.deidentifier.pipelines.testing.PipelineTester - Running test suite with 1 test cases
\end{verbatim}

On a new line add another date without the
\texttt{\textless{}Date\textgreater{}} tags. Run the test suite again
and see how it fails. Add the tags, run again and this time the suite
should pass.

\paragraph{Add missing location}\label{add-missing-location}

The location \texttt{Oberikon} is not recognized in a document. Add a
test case for it in the \texttt{locations.txt} file and run the test
suite. It should fail on that test. Then, add the place to some
appropriate lexikon (e.g.~in the already existing
\texttt{locations/additional\_locations.lst}). After that, the test
suite

\paragraph{Internal phone number
format}\label{internal-phone-number-format}

Assume internal phone numbers consist of two blocks of 3 digits, e.g.:
\texttt{123\ 456}. Write a JAPE rule which recognizes these numbers.
There is already some test case in \texttt{contact.txt}

Hints: * add the rule in \texttt{specific-rules/contacts.jape}. There is
already some rule recognizing some Swiss phone numbers (copied from the
generic JAPE rule set). * See
https://gate.ac.uk/sale/thakker-jape-tutorial/GATE\%20JAPE\%20manual.pdf
if you'd like to know more about how JAPE rules work. * In practice, you
would probably add a ``trigger'' on the left side, i.e.~fire the rule
only if the two blocks are preceded by a ``Tel'' token.
