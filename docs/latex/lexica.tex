\subsection{Lexica}\label{lexica}

In GATE, a lexicon (or gazetteer) consists of a text file with one term
per line. A term may contain spaces. The terms are treated as
case-sensitive.

The lexica compiled for the USZ pipeline have many different sources.
Below tables describing the different lexica along with their
provenance. Note, that some lexica cannot be published, as they contain
hospital internal data. Note, that \texttt{ANNIE} refers to a GATE
plugin, \texttt{GeoNames} to the geographical database which can be
found here \texttt{https://www.geonames.org/}.
\newpage

\subsubsection{General}\label{general}
\begin{longtable}[]{@{}
  >{\raggedright\arraybackslash}p{(\columnwidth - 4\tabcolsep) * \real{0.3405}}
  >{\raggedright\arraybackslash}p{(\columnwidth - 4\tabcolsep) * \real{0.2797}}
  >{\raggedright\arraybackslash}p{(\columnwidth - 4\tabcolsep) * \real{0.3797}}@{}}
\toprule\noalign{}
\begin{minipage}[b]{\linewidth}\raggedright
File Name
\end{minipage} & \begin{minipage}[b]{\linewidth}\raggedright
Source
\end{minipage} & \begin{minipage}[b]{\linewidth}\raggedright
Description
\end{minipage} \\
\midrule\noalign{}
\endhead
\bottomrule\noalign{}
\endlastfoot
abbreviations\_stop.lst & ANNIE German & Abbreviations like
\texttt{z.B} \\
general\_wordlist\_with\_uppercased.lst & Aspell dictionary & General
wordlist \\
stop.lst & ANNIE German & Stopwords \\
\end{longtable}

\subsubsection{Locations}\label{locations}
\paragraph{Geographical}\label{geographical}
\begin{longtable}[]{@{}
  >{\raggedright\arraybackslash}p{(\columnwidth - 4\tabcolsep) * \real{0.3064}}
  >{\raggedright\arraybackslash}p{(\columnwidth - 4\tabcolsep) * \real{0.1611}}
  >{\raggedright\arraybackslash}p{(\columnwidth - 4\tabcolsep) * \real{0.5025}}@{}}
\toprule\noalign{}
\begin{minipage}[b]{\linewidth}\raggedright
File Name
\end{minipage} & \begin{minipage}[b]{\linewidth}\raggedright
Source
\end{minipage} & \begin{minipage}[b]{\linewidth}\raggedright
Description
\end{minipage} \\
\midrule\noalign{}
\endhead
\bottomrule\noalign{}
\endlastfoot
additional\_locations.lst & Manually edited & \\
canton\_names.lst & GeoNames and manually added & ``Uri'' \\
cantons\_abbrevs.lst & GeoNames and manually added & ``ZH'', ``BE'' \\
citizenships.lst & Wikipedia and manually added & ``Schweizer'',
``Deutsche'' \\
city.lst & ANNIE & various cities (worldwide) \\
city\_ambiguous\_manual.lst & Manually edited & Cities which typically
have another meaning in the context of medical reports,
e.g.~\texttt{Wangen}, \texttt{Füssen}. No Location annotations are
performed. \\
city\_derived.lst & ANNIE & \\
city\_german.lst & ANNIE & \\
city\_switzerland.lst & Swisstopo Ortschaftenverzeichnis & \\
country.lst & ANNIE & \\
country\_adjectives\_german.lst & Wikipedia and manually added &
``Italienisch'', ``Italienischer'' \\
country\_german.lst & ANNIE & \\
country\_german\_wiki.lst & Wikipedia & Country Names \\
country\_iso\_codes.lst & Manual & \\
country\_manual.lst & Manual & Contains countries not existing
anymore \\
country\_regions\_german\_wiki.lst & Wikipedia & ``Norditalien'' \\
languages\_manual.lst & Manual & \\
larger\_cities.lst & GeoNames & Larger international cities,
``Tripoli'', ``Hannover'' \\
location\_false\_positives.lst & Manual & Typically medical terms
containing location as a part (not annotated) \\
province.lst & ANNIE & \\
regions.lst & Manual & \\
streetnames.lst & Manual & Streets or similar not matching a typical
pattern \\
toponyms\_switzerland.lst & GeoNames & All sorts of topopynms. Extensive
blacklist was needed for ambiguous locations \\
toponyms\_switzerland\_manual.lst & Manual & More Swiss Toponyms \\
\end{longtable}
\newpage

\paragraph{Organisational}\label{organisational}
\begin{longtable}[]{@{}
  >{\raggedright\arraybackslash}p{(\columnwidth - 4\tabcolsep) * \real{0.3483}}
  >{\raggedright\arraybackslash}p{(\columnwidth - 4\tabcolsep) * \real{0.1461}}
  >{\raggedright\arraybackslash}p{(\columnwidth - 4\tabcolsep) * \real{0.5056}}@{}}
\toprule\noalign{}
\begin{minipage}[b]{\linewidth}\raggedright
File Name
\end{minipage} & \begin{minipage}[b]{\linewidth}\raggedright
Source
\end{minipage} & \begin{minipage}[b]{\linewidth}\raggedright
Description
\end{minipage} \\
\midrule\noalign{}
\endhead
\bottomrule\noalign{}
\endlastfoot
buildings\_usz.lst & USZ (KISIM) & Building abbreviations at USZ \\
hospitals.lst & Manual & Hospital names \\
institutions.lst & Manual & Institutions related to USZ \\
organisational\_units\_usz.lst & USZ (KISIM) & Abbreviations of
organisational units \\
related\_organisations\_usz.lst & USZ (KISIM) & Institutions related to
USZ. Internal only. \\
\end{longtable}

\subsubsection{Medical}\label{medical}

Including information about medical terms into the pipeline is mainly to
avoid an annotation on it, typically for surnames.

\begin{longtable}[]{@{}
  >{\raggedright\arraybackslash}p{(\columnwidth - 4\tabcolsep) * \real{0.2745}}
  >{\raggedright\arraybackslash}p{(\columnwidth - 4\tabcolsep) * \real{0.2013}}
  >{\raggedright\arraybackslash}p{(\columnwidth - 4\tabcolsep) * \real{0.5242}}@{}}
\toprule\noalign{}
\begin{minipage}[b]{\linewidth}\raggedright
File Name
\end{minipage} & \begin{minipage}[b]{\linewidth}\raggedright
Source
\end{minipage} & \begin{minipage}[b]{\linewidth}\raggedright
Description
\end{minipage} \\
\midrule\noalign{}
\endhead
\bottomrule\noalign{}
\endlastfoot
drugs\_usz.lst & USZ (KISIM) & \\
medical\_mesh\_terms.lst & MESH 2019 German Translation &
https://www.dimdi.de/dynamic/en/classifications/further-classifications-and-standards/mesh/ \\
medical\_terms\_de.lst & Wikipedia & \\
medical\_terms\_manual.lst & Manual & \\
\end{longtable}

\subsubsection{Occupations}\label{occupations}

Professions and companies a patient might work for.

\begin{longtable}[]{@{}
  >{\raggedright\arraybackslash}p{(\columnwidth - 4\tabcolsep) * \real{0.2500}}
  >{\raggedright\arraybackslash}p{(\columnwidth - 4\tabcolsep) * \real{0.1100}}
  >{\raggedright\arraybackslash}p{(\columnwidth - 4\tabcolsep) * \real{0.6400}}@{}}
\toprule\noalign{}
\begin{minipage}[b]{\linewidth}\raggedright
File Name
\end{minipage} & \begin{minipage}[b]{\linewidth}\raggedright
Source
\end{minipage} & \begin{minipage}[b]{\linewidth}\raggedright
Description
\end{minipage} \\
\midrule\noalign{}
\endhead
\bottomrule\noalign{}
\endlastfoot
company\_list\_ch.lst & Wikipedia & \\
generic\_occupations.lst & Manual & More for testing purposes \\
occupations\_usz.lst & USZ Kisim & Processed list of professions entered
in KISIM. Internal only. \\
\end{longtable}

\subsubsection{Person Names}\label{person-names}

A few lexica are partitioned into two parts with ``frequent'' and
``seldom'' names. The distinction is used by some rules as indication
whether a token might likely be a name or not.

\begin{longtable}[]{@{}
  >{\raggedright\arraybackslash}p{(\columnwidth - 4\tabcolsep) * \real{0.3274}}
  >{\raggedright\arraybackslash}p{(\columnwidth - 4\tabcolsep) * \real{0.4867}}
  >{\raggedright\arraybackslash}p{(\columnwidth - 4\tabcolsep) * \real{0.1858}}@{}}
\toprule\noalign{}
\begin{minipage}[b]{\linewidth}\raggedright
File Name
\end{minipage} & \begin{minipage}[b]{\linewidth}\raggedright
Source
\end{minipage} & \begin{minipage}[b]{\linewidth}\raggedright
Description
\end{minipage} \\
\midrule\noalign{}
\endhead
\bottomrule\noalign{}
\endlastfoot
firstnames\_switzerland\_frequent.lst & Bundesamt für Statistik:
Vornamen in der Schweiz 2017 & \\
firstnames\_switzerland\_seldom.lst & Bundesamt für Statistik: Vornamen
in der Schweiz 2017 & \\
firstnames\_usz\_frequent.lst & USZ (KISIM) & \\
name\_false\_positives.lst & Manual & Often medical terms \\
surnames\_usz\_frequent.lst & USZ (KISIM) & \\
firstnames\_usz\_seldom.lst & USZ (KISIM) & Internal only \\
surnames\_staff\_usz.lst & USZ (KISIM) & Internal only \\
surnames\_usz\_seldom.lst & USZ (KISIM) & Internal only \\
\end{longtable}

\subsubsection{Suffix Lists}\label{suffix-lists}

Instead of using lexica annotating terms 1:1 in the text, a part of the
pipeline annotates tokens based on suffixes. This is useful for missing
terms in the other dictionaries. For example \texttt{Nasenerkrankung}
would still be recognized as medical term, even though it might not be
in any medical dictionary.

\begin{longtable}[]{@{}lll@{}}
\toprule\noalign{}
File Name & Source & Description \\
\midrule\noalign{}
\endhead
\bottomrule\noalign{}
\endlastfoot
medical\_suffixes.lst & Manual &
\texttt{erkrankung},\texttt{geräusch} \\
surname\_suffixes.lst & Manual & \texttt{mann},\texttt{oulos} \\
\end{longtable}
